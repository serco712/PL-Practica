\chapterA{Especificación formal del léxico}


\section{Definiciones auxiliares.}
    
\begin{math}
    letra \longrightarrow \textbf A|\textbf B|...|\textbf Z|\textbf a|\textbf b|...|\textbf z\\
    digitoPositivo \longrightarrow \textbf 1|...|\textbf 9\\
    digito \longrightarrow digitoPositivo|0\\
    parteEntera \longrightarrow digitoPositivo \space digito*\\
    parteDecimal \longrightarrow digito* \space digitoPositivo\\
    parteExponencial \longrightarrow (e|E)[\backslash{+}|-]parteEntera\\
\end{math}

\section{Definiciones de cadenas ignorables.}
\section{Definiciones léxicas.}

\begin{math}
    bool \longrightarrow \textbf{bool}\\
    int \longrightarrow \textbf{int}\\
    real \longrightarrow \textbf{real}\\
    and \longrightarrow \textbf{and}\\
    or \longrightarrow \textbf{or}\\
    not \longrightarrow \textbf{not}\\
    literalBooleano \longrightarrow \textbf{true}|\textbf{false}\\
    literalEntero \longrightarrow [\backslash{+}|-]parteEntera\\
    literalReal \longrightarrow [\backslash{+}|-]parteEntera(.parteDecimal|parteExponencial|.parteDecimal parteExponencial)\\
    identificador \longrightarrow (\_|letra)(letra|digito|\_)*\\
    operadorSuma \longrightarrow \backslash{+}\\
    operadorResta \longrightarrow -\\
    operadorMul \longrightarrow \backslash{*}\\
    operadorDiv \longrightarrow /\\
    operadorMenor \longrightarrow  <\\
    operadorMayor \longrightarrow  >\\
    operadorIgual \longrightarrow  ==\\
    operadorMenIgual \longrightarrow  <=\\
    operadorMayIgual \longrightarrow  >=\\
    operadorAsig \longrightarrow  =\\
    parentesisAp \longrightarrow  \backslash{(}\\
    parentesisCi \longrightarrow  \backslash{)}\\
    puntoYComa \longrightarrow  ;\\

\end{math}