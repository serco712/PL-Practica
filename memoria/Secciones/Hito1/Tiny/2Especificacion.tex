\section{Especificación formal del léxico}


\subsection{Definiciones auxiliares.}
    
\begin{math}
    letra \longrightarrow \textbf A|\textbf B|...|\textbf Z|\textbf a|\textbf b|...|\textbf z\\
    digitoPositivo \longrightarrow \textbf 1|...|\textbf 9\\
    digito \longrightarrow digitoPositivo|0\\
    parteEntera \longrightarrow digitoPositivo \space digito*\\
    parteDecimal \longrightarrow digito* \space digitoPositivo\\
    parteExponencial \longrightarrow (e|E)[\backslash{+}|-]parteEntera\\
\end{math}

\subsection{Definiciones de cadenas ignorables.}

\begin{math}
    separador \longrightarrow \textbf{SP}|\textbf{TAB}|\textbf{NL}\\
    comentario \longrightarrow \#\#(\overline{\textbf{NL}|\textbf{EOF}})\\
\end{math}

\subsection{Definiciones léxicas.}

\begin{math}
    bool \longrightarrow \textbf{bool}\\
    int \longrightarrow \textbf{int}\\
    real \longrightarrow \textbf{real}\\
    string \longrightarrow \textbf{string}\\
    and \longrightarrow \textbf{and}\\
    or \longrightarrow \textbf{or}\\
    not \longrightarrow \textbf{not}\\
    null \longrightarrow \textbf{null}\\
    proc \longrightarrow \textbf{proc}\\
    if \longrightarrow \textbf{if}\\
    else \longrightarrow \textbf{else}\\
    while \longrightarrow \textbf{while}\\
    struct \longrightarrow \textbf{struct}\\
    new \longrightarrow \textbf{new}\\
    delete \longrightarrow \textbf{delete}\\
    read \longrightarrow \textbf{read}\\
    write \longrightarrow \textbf{write}\\
    nl \longrightarrow \textbf{nl}\\
    type \longrightarrow \textbf{type}\\
    call \longrightarrow \textbf{call}\\
    literalBooleano \longrightarrow \textbf{true}|\textbf{false}\\
    literalEntero \longrightarrow [\backslash{+}|-]parteEntera\\
    literalReal \longrightarrow [\backslash{+}|-]parteEntera(.parteDecimal|parteExponencial|.parteDecimal parteExponencial)\\
    literalCadena \longrightarrow \textbf{literalCadena}\\
    identificador \longrightarrow (\_|letra)(letra|digito|\_)*\\
    operadorSuma \longrightarrow \backslash{+}\\
    operadorResta \longrightarrow \; -\\
    operadorMul \longrightarrow \; \backslash{*}\\
    operadorDiv \longrightarrow \; /\\
    operadorMod \longrightarrow \%\\
    operadorMenor \longrightarrow \; <\\
    operadorMayor \longrightarrow \; >\\
    operadorIgual \longrightarrow \; ==\\
    operadorMenIgual \longrightarrow \; <=\\
    operadorMayIgual \longrightarrow \; >=\\
    operadorAsig \longrightarrow \; =\\
    parentesisAp \longrightarrow \; \backslash{(}\\
    parentesisCi \longrightarrow \; \backslash{)}\\
    puntoYComa \longrightarrow \; ;\\
    arroba \longrightarrow \; @\\
    coma \longrightarrow \; ,\\
    indireccion \longrightarrow \; \hat{}\\
    final \longrightarrow \&\&\\
    porReferencia \longrightarrow \&\\
    corcheteAp \longrightarrow \; \{\\
    corcheteCi \longrightarrow \; \}\\
    corcheteTamAp \longrightarrow \; \[\\
    corcheteTamCi \longrightarrow \; \]\\
    arroba \longrightarrow \; @\\

\end{math}