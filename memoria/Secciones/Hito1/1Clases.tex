\chapterA{Clases léxicas}

\section{Palabras reservadas}

Para poder analizar de manera correcta, será necesario establecer una clase léxica por cada palabra reservada. En el lenguaje de
esta práctica, \textit{Tiny(0)}, contamos con 3 palabras reservadas, utilizadas para definir el tipo de las variables. Tendremos pues,
una palabra para las variables de tipo booleano, otra para las de tipo entero y una última para las reales. Las palabras son 
las definidas a continuación, contando cada con una clase léxica.

\begin{itemize}
    \item \textit{bool} $\rightarrow$ Variables booleanas.
    \item \textit{int} $\rightarrow$ Variables enteras.
    \item \textit{real} $\rightarrow$ Variables reales.
    \item \textit{and} $\rightarrow$ Conjunción lógica.
    \item \textit{or} $\rightarrow$ Disyunción lógica.
    \item \textit{not} $\rightarrow$ Negación lógica.
\end{itemize}

\section{Literales}

\begin{itemize}
    \item \textbf{Literales booleanos.}
    \item \textbf{Literales enteros.}
    \item \textbf{Literales reales.}
\end{itemize}

\section{Identificadores}

\section{Símbolos de operación y puntuación}

\begin{itemize}
    \item \textbf{Suma.}
    \item \textbf{Resta.}
    \item \textbf{Multiplicación.}
    \item \textbf{División.}
    \item \textbf{Menor.}
    \item \textbf{Mayor.}
    \item \textbf{Igual.}
    \item \textbf{Menor o igual.}
    \item \textbf{Mayor o igual.}
    \item \textbf{Asignación.}
    \item \textbf{Paréntesis de apertura.}
    \item \textbf{Paréntesis de cierre.}
    \item \textbf{Punto y coma.}
\end{itemize}
