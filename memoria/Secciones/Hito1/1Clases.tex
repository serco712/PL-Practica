\chapterA{Clases léxicas}

\section{Palabras reservadas}

Para poder analizar de manera correcta, será necesario establecer una clase léxica por cada palabra reservada. En el lenguaje de
esta práctica, \textit{Tiny(0)}, contamos con 3 palabras reservadas, utilizadas para definir el tipo de las variables. Tendremos pues,
una palabra para las variables de tipo booleano, otra para las de tipo entero y una última para las reales. Las palabras son 
las definidas a continuación, contando cada con una clase léxica.

\begin{itemize}
    \item \textit{bool} $\rightarrow$ Variables booleanas.
    \item \textit{int} $\rightarrow$ Variables enteras.
    \item \textit{real} $\rightarrow$ Variables reales.
    \item \textit{and} $\rightarrow$ Conjunción lógica.
    \item \textit{or} $\rightarrow$ Disyunción lógica.
    \item \textit{not} $\rightarrow$ Negación lógica.
\end{itemize}

\section{Literales}

\begin{itemize}
    \item \textbf{Literales booleanos.} 
    \item \textbf{Literales enteros.} Opcionalmente,empiezan con un signo + o -, después debe aparecer una secuencia (que empieza por un número distinto de 0) de 1 o más dígitos 
    \item \textbf{Literales reales.}
\end{itemize}

\section{Identificadores}

\section{Símbolos de operación y puntuación}

\begin{itemize}
    \item \textbf{Suma.} Se representa con el símbolo +
    \item \textbf{Resta.}Se representa con el símbolo símbolo -
    \item \textbf{Multiplicación.} Se representa con el símbolo *
    \item \textbf{División.} Se representa con el símbolo /
    \item \textbf{Menor.} Se representa con el símbolo <
    \item \textbf{Mayor.} Se representa con el símbolo >
    \item \textbf{Igual.} Se representa con el símbolo ==
    \item \textbf{Menor o igual.} Se representa con el símbolo <=
    \item \textbf{Mayor o igual.} Se representa con el símbolo >=
    \item \textbf{Asignación.} Se representa con el símbolo = 
    \item \textbf{Paréntesis de apertura.} Se representa con el símbolo (
    \item \textbf{Paréntesis de cierre.} Se representa con el símbolo )
    \item \textbf{Punto y coma.} Se representa con el símbolo ;
\end{itemize}
